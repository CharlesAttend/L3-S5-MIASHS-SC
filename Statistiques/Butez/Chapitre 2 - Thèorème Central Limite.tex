\documentclass{article}
\usepackage[utf8]{inputenc}
\usepackage[a4paper, margin=2.5cm]{geometry}
\usepackage{graphicx}
\usepackage[french]{babel}

\usepackage[default,scale=0.95]{opensans}
\usepackage[T1]{fontenc}
\usepackage{amssymb} %math
\usepackage{amsmath}
\usepackage{amsthm}
\usepackage{systeme} %use with \system*{eq1, eq2}
\usepackage{bbm}

\usepackage{import}
\usepackage{xifthen}
\usepackage{pdfpages}
\usepackage{transparent}



\newcommand{\incfig}[2][1]{%
    \def\svgwidth{#1\columnwidth}
    \import{./figures/}{#2.pdf_tex}
}

\usepackage{hyperref}
\hypersetup{
    colorlinks=true,
    linkcolor=blue,
    filecolor=magenta,      
    urlcolor=cyan,
    pdftitle={Overleaf Example},
    %pdfpagemode=FullScreen,
}
\urlstyle{same} %\href{url}{Text}

\theoremstyle{plain}% default
\newtheorem{thm}{Théorème}[section]
\newtheorem{lem}[thm]{Lemme}
\newtheorem{prop}[thm]{Proposition}
\newtheorem*{cor}{Corollaire}
%\newtheorem*{KL}{Klein’s Lemma}

\theoremstyle{definition}
\newtheorem{defn}{Définition}[section]
\newtheorem{exmp}{Exemple}[section]
%\newtheorem{xca}[exmp]{Exercise}

\theoremstyle{remark}
\newtheorem*{rem}{Remarque}
\newtheorem*{note}{Note}
%\newtheorem{case}{Case}



\title{Cours}
\author{Charles Vin}
\date{Date}

\begin{document}
\maketitle

\section{Introduction}
\begin{exmp}[]
    Soit $ (X_i)_{i \in \mathbb{N}} $ une suite de v.a. i.i.d., admettant un moment d'ordre 2. \\
    En particulier, les $ X_i $ admettent aussi un moment d'ordre 1. Ainsi d'après la loi forte des grands nombres 
    \[
        \bar{X_n} = \frac{1}{n}\sum_{i=1}^{n}X_i \to^{p.s}_{n \to +\infty}E(X_1)
    .\]
    Pour $ n $ fixé, comment quantifier $ \bar{X_n} $ proche de $ E(X_1) $. "Avec n fixé très grand, on sais que c'est autout de l'espérance mais comment ?"
    Si je reformule, quelle est la taille typique de la variable $ \bar{X_n} - E(X_1) $ \begin{itemize}
        \item Espérance ? : $ E(\bar{X_n}-E(X_1)) = E(\bar{X_n}) - E(X_1) = E(X_1) - E(X_1) = 0 $ 
        \item Variance ? : On pose $ Y_n = \bar{X_n} - E(X_1) $ \begin{align*}
            Var(Y_n) = E((Y_n - E(Y_n))^2) = E(Y_n^2) = Var(\bar{X_n}) = \frac{Var(X_1)}{n}
        \end{align*} 
        $\rightarrow$ Ma variance est petite
    \end{itemize}
    $ Y_n $ est une variable d'espérance 0 et de variance $ \frac{Var(X_1)}{n} $. Par quoi fait-il multiplier $ Y_n $ pour que sa variance soit d'ordre 1 (volontairement impréci, je crois que ça veut dire fini et de grandeur humaine). \\
    Si on regarde $ \sqrt{n} Y_n $ que voit on ? 
    \[
        Var(\sqrt[]{n}Y_n) = n* Var(Y_n) = n * \frac{Var(X_1)}{n} = Var(X_1)
    .\]
    La variance ne dépend plus de $ n $ ! Conclusion : Pour étudier les fluctuation de $ \bar{X_n} $ autour de sa limite $ E(X_1) $, il semble raisonnable d'étudier $ Y_n = \sqrt[]{n}(\bar{X_n} - E(X_1)) $.
\end{exmp}
\begin{defn}[]
    Un variable $ X $ est dit centré et réduite si \begin{itemize}
        \item Elle admet un moment d'ordre 2 
        \item Si $ E(X) = 0, Var(X) = 1 $ 
    \end{itemize}
    Si $ X $ est une variable admettant un moment d'ordre 2, la variable centré réduite issue de $ X $ est $ Z = \frac{X-E(X)}{\sqrt[]{Var(X)}}$ \\
    La centrée réduite de $ \bar{X_n} $ est 
    \[
        Z_n = \frac{\sqrt[]{n} (\bar{X_n}-E(X_1))}{\sqrt[]{Var(X_1)}}
    .\]
    \textbf{Preuve importante à faire en exercice}
\end{defn}

\section{Notion de convergence en loi}
\begin{defn}[Convergence en loi]
    Soit $ (X_n)_{n \in \mathbb{N}} $ une suite de variables aléatoires et $ X $ une variable aléatoire. \\
    On dit que $ (X_n)_{n \in \mathbb{N}}  $ converge en loi vers $ X $ si $ \forall t  $ où $ F_X $ est continue en $ t $ 
    \[
        F_{X_n}(t) \to_{n \to +\infty } F_X(t)
    .\]
    On le note 
    \[
        X_n \to _{n \to +\infty}^{\mathcal{L}} X
    .\]
\end{defn}
\begin{rem}[]
    \begin{itemize}
        \item Ca veut dire quoi $ \{t \in \mathbb{R} \text{ tq } F_X \text{ est continue en } t\} $ ? \\
        $ F_X $ est continue en $ t \Leftrightarrow P(X=t)=0$ 
        \begin{itemize}
            \item Si $ X $ est à densité, $ F_X $ est continue sur $ \mathbb{R} $ 
            \item Si $ X $ est discrète, $ F_X $ est discontinue en chaque point tel que $ P(X=k) > 0 $ 
        \end{itemize}
        \item "Cette notion de convergence est foireuse" Cette notion de convergence n'identifie \textbf{que} la loi de $ X $ \\
        Si on a $ X_n \to^\mathcal{L} X $ et $ Y\perp X  $ de même loi que $ X $ alors $ X_n \to^\mathcal{L} Y$ \begin{proof}[Preuve: ]
            Soit $ t \in \mathbb{R} $ tel que $ F_Y $ est continue en $ t $ . Comme $ F_X = F_Y $, $ F_X $ est aussi continue en $ t $ , et on a comme $ X_n \to ^\mathcal{L} X $ 
            \[
                F_{X_n}(t) \to F_X(t)=F_Y(t)
            .\]
            donc $ X_n \to ^\mathcal{L} Y $ 
        \end{proof}
    \item La remarque 2 implique  : si $ X_n \to^\mathcal{L} X $ et $ Y_n \to^\mathcal{L} Y $  
    \[
        \lim_{n \to \infty} X_n + Y_n \neq X+Y
    .\]
    Exemple : $ X_n, Y_n, X \sim \mathcal{N}(0,1) $ 
    \[
        X_n \to^\mathcal{L}X \text{ car } \forall n, t, F_{X_n} (t) = F_X(t)
    .\]
    \[
        Y_n \to ^\mathcal{L} -X
    .\]
    \[
        \forall t \in \mathbb{R}, F_{Y_n}(t)=F_{X}(t) = F_{-X}(t)
    .\]
    Comme la gaussienne est symetrique $ -X $ est une gaussienne également ect \\
    Mais on a \textbf{pas} $ X_n + Y_n \to^\mathcal{L}X_X=0  $ FAUX car $ X_1 $ indép $ Y_1 $ , $ X_n + Y_n \sim \mathcal{N}(0,2) $ 
    \end{itemize}
\end{rem}

\begin{prop}[(admise)]
    Un suite de v.a. $ (X_n)_{n \in \mathbb{N}} $ converge en loi vers une variable aléatoire $ X $ ssi 
    \[
        \forall f \in C^o_b (\mathbb{R}), E(f(X_n)) \to E(f(X))
    .\]
    Exercice : En admettant cette proposition, montrer que si $ X_n \to ^\mathcal{L}X $ et si $ h \in C^o(\mathbb{R}) $ alors $ h(X_n) \to ^\mathcal{L} h(X) $ 
\end{prop}
\textbf{Méthode : } \begin{enumerate}
    \item Calculer $ F_X $ 
    \item Déterminer $ A=\{t \in \mathbb{R}, F_x \text{ est } C^o \text{ en } t\} $ 
    \item Calculer $ F_{X_n}(t) $ pour $ t \in A $ et montrer $ F_{X_n}(t) \to F_X(t) $ 
\end{enumerate}
Autre cas : \\
Si je vous donne $ (X_n)_{n \in \mathbb{N}} $ et que je vous demande si $ (X_n)_{n \in \mathbb{N}} $ converge en loi. Que faut-il faire ? \begin{itemize}
    \item Je calcule $ F_{X_n}(t), \forall t \in \mathbb{R}$.
    \item Je montre $ F_{X_n}(t) \to_{n \to +\infty } g(t)$ (pas forcement pour tous les $ t $ )
    \item Je dois reconnaitre une variable $ Z $ tq $ F_Z = g $ 
    \item Il faudra avoir montré la convergence pour tous les t tel que $ F_Z $ est continue en $ t $.
\end{itemize}
\begin{exmp}[]
    Soit $ (pn) $ une suite qui converge vers $ p \in ]0,1[ $. $ X_n $ v.a. de loi $ \mathcal{B}er(pn) $ \\
    Montrons que si $ X $ est une v.a. de loi $ \mathcal{B}er(p) $ on a 
    \[
        X_n \to ^\mathcal{L}X
    .\]
    \begin{enumerate}
        \item Soit $ t \in \mathbb{R} $
        \[
            F_X(t) = P(X \leq t) = \systeme*{
                0 \text{ si } t<0,
                1-p \text{ si } t \in [0;1[,
                1 \text{ si } t \geq 1
            }
        .\]
        \item $ F_X $ est continue en t sur $ R \setminus \{0,1\} $ 
        \item Soit $ t \in R \setminus \{0,1\}$ 
        \[
            F_{X_n}(t) = \systeme*{
                0 \text{ si } t<0,
                1 - pn \text{ si } t \in [0;1[,
                1 \text{ si } t \geq 1
            }
        .\]
        Si $ t < 0 $, alors 
        \[
            F_{X_n} = 0 \to 0=F_X(t)
        .\]
        Si $ t \in ]0,1[ $ 
        \[
            F_{X_n}(t) = 1-p_n \to 1-p = F_X(t) \text{ car } p_n \to p
        .\]
        Si $ t>1 $ 
        \[
            F_{X_n}(t) = 1 \to 1 = F_X(t) 
        .\]
        Ainsi, $ \forall t \in \mathbb{R}\setminus \{0,1\}, F_{X_n}(t) \to F_X(t) $ donc $ X_n \to ^\mathcal{L}X $ 
    \end{enumerate}
\end{exmp}
\begin{exmp}[]
    $ \lambda _n \to \lambda, \forall \lambda _n>0, \lambda >0 $. Soit $ (X_n)_{n \in \mathbb{N}} $  suite de v.a. de loi $ \mathcal{E}(\lambda _n) $ et $ X $ une v.a. de loi $ \mathcal{E}(\lambda ) $. Montrer que $ X_n \to ^\mathcal{L}X $ \begin{enumerate}
        \item Soit $ t \in \mathbb{R} $ 
        \[
            F_X(t) = \systeme*{
                0 \text{ si } t<0,
                \int_{0}^{t}\lambda e^{-\lambda x}dx \text{ si } t \geq 0
            } = \systeme*{
                0 \text{ si } t \leq 0,
                {(1-e^{-\lambda t})} \text{ si } t \geq 0
            }
        .\]
        \item $ F_X $ est continue sur $ \mathbb{R} $ ($ X $ est à densité)
        \item Soit $ T \in \mathbb{R} $ 
        \[
            F_{X_n}(t)= \systeme*{
                0 \text{ si } t \leq 0, 
                ({1-e^{-\lambda nt})} \text{ si } t \geq 0
            }
        .\]
        Si $ t \leq 0 $, $ F_{X_n}(t) = 0 \to 0=F_X(t) $ \\
        Si $ t > 0, F_{X_n}(t) = (1-e^{-\lambda nt}) \to {(1-e^{-\lambda t})} = F_X(t) $ car $ \lambda _n \to \lambda  $ et l'expo est continue \\
        Ainsi, $ \forall t \in \mathbb{R}, F_{X_n}(t) \to F_X(t) $ donc $ X_n \to ^\mathcal{L} X $ 
    \end{enumerate}
\end{exmp}

\underline{Nouveau cours du 21/10} \\

\section{Le Théorème Central Limite}
\begin{thm}[Théorème Central Limite]
    Soit $ (X_n)_{n \in \mathbb{N}} $ une suite de variable aléatoire \textbf{i.i.d.} admettant un \textbf{moment d'ordre 2}.

    Soit $ m = E(X_i), \sigma = \sqrt[]{Var(X_i)} $ alors 
    \[
        Z_n = \frac{\sqrt[]{n}}{\sigma }(\frac{1}{n}\sum_{i=1}^{n}X_i-m) \to_{n \to +\infty }^{\mathcal{L}} Z \sim \mathcal{N}(0,1)
    .\]
    Cela signifie : 
    \[
        \forall t \in \mathbb{R}, P(Z_n \leq t) \to_{n \to \infty }\int_{-\infty }^{t}\frac{1}{\sqrt[]{2 \pi }}e^{-\frac{x^2}{2}}dx
    .\]
\end{thm}
\begin{rem}[]
    $ n=10, X_n $ i.i.d. uniforme sur $ [0,1], Var X_i = 1/12, \sigma = \frac{1}{\sqrt[]{12}} $
    \begin{align*}
        Z{_10} &= \frac{\sqrt[]{10}}{1/\sqrt[]{12}}(\frac{1}{10} \sum_{i=1}^{10}X_i - \frac{1}{2}) \\
            &= \sqrt[]{120}(\frac{1}{10} \sum_{i=1}^{10}X_i - 1/2) 
    \end{align*}
    Comment voir la loi de $ Z{_10} $ ? \\
    $ Z{_10}^\prime, \dots, Z{_10}^{(N)} $ copies i.i.d. de $ Z{_10} $. \\
    Glivenko-Conteli nous assure que 
    \[
        t \in \mathbb{R} \sup \left| \frac{1}{N} \sum_{i=1}^{N} \mathbbm{1}_{Z{_10}^{(i)} \leq t} - F_{Z{_10}}(t)\right| \to_{n \to \infty }^{p.s} 0
    .\]
    C'est bien mais super visuel.

    A la place, on préfère tracer des histogrammes (voir le cours de Mma. Fradon). On regarde si l'histogrammes suit la densité d'une $ \mathcal{N}(0,1) $ 
\end{rem}
On note $ \phi (t) = \int_{_\infty }^{t}\frac{e^{-x^2 /2}}{\sqrt[]{2 \pi }} $ 
\begin{prop}[]
    Soit les hypothèse précédantes, on a 
    \[
        \forall a,b \in \mathbb{R}, P(a \leq Z_n \leq b) \to P(a \leq Z \leq b) = \int_{a}^{b}\frac{e^{-x^2 /2}}{\sqrt[]{2 \pi }} = \phi (b) - \phi (a)
    .\]
    De même pour \begin{align*}
        P(a < Z_n < b) \to \phi (b) - \phi (a) \\
        P(a \leq Z_n < b) \to \phi (b) - \phi (a) \\
        P(a < Z_n \leq  b) \to \phi (b) - \phi (a)
    \end{align*}

    \begin{proof}[Preuve: ]
        Uniquement pour $ a < Z_n \leq b $

        Soit $ a,b \in \mathbb{R} $ 
        \[
            P(a < Z_n \leq b) = F_{Z_n}(b) - F_{Z_n}(a) \text{ car } P(Z_n \leq b) - P(Z_n \leq a)
        .\]
        Or d'après le théorème central limite 
        \[
            F_{Z_n}(b) \to \phi (b)
        .\]
        De même 
        \[
            F_{Z_n}(a) \to \phi (a)
        .\]
        Donc : 
        \[
            P(a < Z_n \leq b) = F_{Z_n}(b) - F_{Z_n}(a) \to \phi(b)- \phi (a)
        .\]
        Pour tous les autres cas, ce n'est pas immédiat. Cela provient de la continuité de $ \phi  $ 
    \end{proof}
    
    
\end{prop}

\begin{defn}[Intervalle de confiance asymptotique]
    Soit $ (\Omega , \mathcal{F}, (P_\theta )_{\theta \in \Theta }) $ un modèle statistique. Soit $ X_1, \dots, X_n $ un échantillon d'une loi $ P_\theta  $. 

    Un intervalle de confiance asymptotique de niveau $ 1 - \alpha  $ pour une quantité $ g(\theta ) $ est un intervalle aléatoire $ I(X_1, \dots, X_n) $, ne dépendant pas de $ \theta  $ et tel que 
    \[
        \forall \theta \in \Theta, \lim_{n \to \infty}  P(g(\theta ) \in I(X_1, \dots, X_n)) \geq 1-\alpha 
    .\]
\end{defn}
\begin{exmp}[\textbf{Application}]
    Détermination d'un intervalle de confiance asymptotique à l'aide du théorème centrale limite 

    Soit $ (X_i)_{i \in \mathbb{N}} $ une suite de v.a. i.i.d. d'espérance inconnue et de variance égale à 35. 

    On cherche à estimer $ m $ et à donner un IC. asymptotique.

    Soit $ \bar{X_n} = \frac{1}{n}\sum_{i=1}^{n}X_i $. Comme $ (X_i)_{i \in \mathbb{N}} $ est une suite de v.a. i.i.d. admettant un moment d'ordre 2, d'après le théorème central limite, on a 
    \[
        Z_n = \frac{\sqrt[]{n}}{\sqrt[]{35}}(\bar{X_n}-m) \to_{n \to +\infty }^\mathcal{L} Z \sim \mathcal{N}(0,1)
    .\]
    Ainsi 
    \[
        \forall a \in \mathbb{R}, P(-a \leq Z_n \leq a) \to_{n \to \infty } \int_{-a}^{a}\frac{e^{-x^2 /2}}{\sqrt[]{2 \pi }}dx = \phi (a) - \phi (-a)
    .\]
    Si je souhaite un intervalle de confiance asymptotique de niveau 99\%. \\
    On cherche $ a $ tel que : $ \phi(a) - \phi (-a) \geq 0.99 $ \\
    On cherche $ a $ tel que : $ P(Z \geq a) = 0.005 $ (l'aire restante à droite de la gaussienne) $ P(Z \geq a) = 1 - \phi (a) = 0.005 $.
    Ainsi, on cherche $ a \in \mathbb{R} $ tel que $ \phi (a) = 0.995 $. Lecture de la table : si $ a=2.58, \phi (a) = 0.995 $. Ainsi, $ P(-2.58 \leq Z \leq 2.58) \geq 0.99 $ On a donc :
    \[
        \lim_{n \to \infty} P(-2.58 \leq \frac{\sqrt[]{n}}{\sqrt[]{35}} (\bar{X_n} - m ) \leq 2.58) \geq 0.99
    .\]
    Ré exprimons $ -2.58 \leq Z_n \leq 2.58 $ en $ m \in I(X_1, \dots, X_n) $
    \begin{align*}
        -2.58 \leq \frac{\sqrt[]{n}}{\sqrt[]{35}} (\bar{X_n} - m ) \leq 2.58 \\
        \frac{2.58 * \sqrt[]{35}}{\sqrt[]{n}} \geq m - \bar{X_n} \geq \frac{-2.58* \sqrt[]{35}}{\sqrt[]{n}} \\
        m \in [\bar{X_n} - \frac{2.58 * \sqrt[]{35}}{\sqrt[]{n}} , \bar{X_n} + \frac{2.58 * \sqrt[]{35}}{\sqrt[]{n}}]\\
        2.58 * \sqrt[]{35} \approx 15.26
    \end{align*}
    On obtient alors 
    \[
        \lim_{n \to \infty} P(m \in [\bar{X_n} \pm  \frac{15.26}{\sqrt[]{n}}]) \geq 0.99
    .\]
    Un intervalle de confiance asymptotique de niveau 99\% est 
    \[
        [\bar{X_n} \pm \frac{2.56 * \sqrt[]{35}}{\sqrt[]{n}}]
    .\]
\end{exmp}


\underline{Nouveau cours du 28/10} \\

\section{Précision du théorème central limite }
Le TCL stipule que 
\[
    \forall a \in \mathbb{R}, P(Z_n \leq a) \to_{n \to +\infty } P(Z \leq a) = \phi (a)
.\]
Ou bien : 
\[
    \forall a \in \mathbb{R}, P(-a \leq Z_n \leq a) \to _{n \to \infty } \phi (a) - \phi (-a)
.\]
Malheuresement, à ce stade, nous n'avons aucune idée de la valeur de $ \left| P(Z_n \leq a) - \phi (a) \right|  $ pour une valeur de $ n $ fixée. \\
Pour $ a $ fixé, quelle est la qualité de cette approximation ? Pour mesurer cette différence, je souhaite controler 
\[
    \sup_{a \in \mathbb{R}} \left| F_{Z_n}(a) - \phi (a) \right| 
.\]
\begin{note}[]
    C'est la distance entre les deux fonctions de répartion dans le pire des cas
\end{note}
\begin{rem}[]
    Cela n'a RIEN à voir avec le théorème de Glivenko-Cantelli
\end{rem}

\begin{thm}[Inégalité de Berry-Esseen]
    Soit $ (X_i)_ {i \in \mathbb{N}} $ une suite de v.a. i.i.d., admettant un moment d'ordre 3.

    On note  $ m = E(X_1), \sigma =\sqrt[]{Var(X_1)}, \rho = E( \left| X_1-m \right| ^3), Z_n = \frac{\sqrt[]{n}}{\sigma } ((\frac{1}{n}\sum_{i=1}^{n}X_i) - m)$ on a :
    \[
        \forall a \in \mathbb{R}, \left| F_{Z_n}(a) - \phi (a) \right| \leq \frac{1}{2}\frac{\rho }{\sigma ^3 \sqrt[]{n}}
    .\]
\end{thm}
\begin{rem}[]
    Pour pouvoir utiliser ce théorème, il faut connaitre la valeur exacte de $ \sigma  $ et $ \rho $. Calculer $ \rho = E(\left| X-m \right| ^3) $ 
    \begin{exmp}[]
        Si $ X \sim \mathcal{B}(p), m=p $ \begin{align*}
            E( \left| X-p \right| ^3) &= P(X=0) p^3 + P(X=1) \left| 1-p \right| ^3 \\
                            &= (1-p)p^3 + p(1-p)^3
        \end{align*}
        Si $ X \sim \mathcal{E}(\lambda ), m=1/\lambda  $ 
        \[
            E(\left| X - \frac{1}{\lambda } \right| ^3) = \int_{0}^{+\infty }\left| x-\frac{1}{\lambda } \right| ^3 \lambda e^{-\lambda x}dx
        .\]
    \end{exmp}
\end{rem}

\textbf{Comment se servir de ce théorème ?} Application aux intervalles de confiance.

$ X_1, \dots, X_n $ v.a. i.i.d. d'esperance $ m $ inconnue, mais admettant un moment d'ordre 3, avec $ \sigma = \sqrt[]{Var(X_1)} $ et $ \rho =E(\left| X_1 - m \right| ^3) $ connus. \\
D'après le théorème central limite, appliqué à la suite $ (X_i)_{i \in \mathbb{N}} $ 
\[
    \forall a \in \mathbb{R}, P(Z_n \leq a) \to _{n \to +\infty } \phi (a)
.\]
Or d'après l'inégalité de Berry-Esseen, on a 
\[
    \left| P(Z_n \leq a) - \phi (a) \right| \leq \frac{1}{2}\frac{\rho }{\sigma ^3}\sqrt[]{n}
.\]
Cela donne: 
\[
    \phi (a) - \frac{1}{2}\frac{\rho }{\sigma ^3 \sqrt[]{n}}\leq P(Z_n \leq a) \leq \phi (a) + \frac{1}{2}\frac{\rho }{\sigma ^3 \sqrt[]{n}}
.\]
Si on veut $ a $ tq $ \phi (a) = 0.95 \Leftrightarrow a = 1.65 $ alors 
\[
    0.95 - \frac{1}{2}\frac{\rho }{\sigma ^3 \sqrt[]{n}} \leq P(Z_n \leq 1.65) \leq 0.95 + \frac{1}{2}\frac{\rho }{\sigma ^3 \sqrt[]{n}}
.\]




On change d'exemple  : Si on regarde $ P(-a \leq Z_n \leq a) \approx \phi (a) - \phi (-a) $ on regarde en faite $ F_{Z_n}(a) - F_{Z_n}(-a) \approx \phi (a) - \phi (-a) $ .\\
Si on choisit $ a $ tq $ \phi (a) - \phi (-a) = 0.96 \Leftrightarrow \phi (a) = 0.98 \Leftrightarrow a \approx 2.06$

D'après le TCL, on avait 
\[
    P(-2.06 \leq Z_n \leq 2.06) \to_{n \to +\infty } 0.96
\]
\[
    P(m \in (\bar{X_n} - \frac{2.06 \sigma }{\sqrt[]{n}}, \bar{X_n} + \frac{2.06 \sigma }{\sqrt[]{n}}))
.\]
\begin{align*}
    P(-a \leq Z_n \leq a) &= P(Z_n \leq a) - P(Z_n < -a) \to \phi (a) - \phi (-a) \\ 
    &\Leftrightarrow \left| P(-a \leq Z_n \leq a) - (\phi (a) - \phi (-a))\right| \\ 
    &= \left| P(Z_n \leq a) - P(Z_n < -a) - (\phi (a) - \phi (-a)) \right| \\
    &= \left| (P(Z_n \leq a) - \phi (a)) - (P(Z_n < -a) - \phi (-a)) \right| \\
    & \leq \left| P(Z_n \leq a) - \phi (a) \right| + \left| P(Z_n < -a) - \phi(-a) \right|
\end{align*}
D'après l'inégalité de Berry-Esseen, on a : \begin{align*}
    \left| P(-a \leq Z_n \leq a) - (\phi (a) - \phi (-a)) \right| & \leq \frac{\phi }{2 \sigma ^3 \sqrt[]{n}} + \frac{\rho }{2 \sigma ^3 \sqrt[]{n}} \\ 
        & \leq \frac{\rho }{\sigma ^3 \sqrt[]{n}}
\end{align*}
Pour $ a=2.06 $ cela donne  
\[
    0.96 - \frac{\rho }{\sigma ^3 \sqrt[]{n}} \leq P(-2.06 \leq Z_n \leq 2.06) \leq 0.96 + \frac{\rho }{\sigma ^3 \sqrt[]{n}}
.\]
Pour $ \rho =10, \sigma = 1, n = 10000 $ 
\[
    \frac{\rho }{\sigma ^3 \sqrt[]{n}} = \frac{10}{100} = 0.1, P(-2.06 \leq Z_n \leq 2.06) \geq 0.86
.\]
L'inégalité de Berry-Esseen permet de passer d'un intervalle de confiance asymptotique de niveau $ 1-\alpha  $ à un intervalle de confiance \textbf{exact} de niveau $ 1-\alpha  $ avec $ \alpha ^\prime  $ explicite. $ \alpha ^\prime  = \alpha - \frac{rho
}{\sigma ^3 \sqrt[]{n}}$ \\
Vous pouvez comparer avec Tchebychev. Le meilleur dépendra des valeur de $ \rho ,n, \sigma  $. En général, $ n $ grand $\rightarrow$ TCL + Berry-Esseen // $ n $ petit $\rightarrow$ Tchebychev.

Si on veut garantir un niveau fixé, on peut anticiper. C'est à dire choisir $ \alpha  $ tq, par exemple, $ 1- \alpha \geq 0.95$. (Quand c'est possible, je ne peux pas toujours anticiper car $ \alpha \geq 0 $ ). 
Reprenons l'exemple : \\
On sait 
\[
    (\phi (a) - \phi (-a)) - \frac{\rho }{\sigma ^3 \sqrt[]{n}} \leq P(-a \leq Z_n \leq a)
.\]
On cherche alors $ a $ tq 
\begin{align*}
    &\phi (a) - \phi (-a) - \frac{\rho }{\sigma ^3 \sqrt[]{n}} = 1 - \alpha  \\
    & \Leftrightarrow \phi (a) - \phi (-a) = 1 - \alpha + \frac{\rho }{\sigma ^3 \sqrt[]{n}}\\
    & \Leftrightarrow \phi (a) = 1 - \frac{1}{2}(\alpha - \frac{\phi }{\sigma ^3 \sqrt[]{n}}) = 1 - \frac{\alpha }{2} + \frac{\phi }{\sigma ^3 \sqrt[]{n}}
\end{align*} 
Il est possible, si $ 1 - \frac{\alpha }{2} + \frac{\phi }{\sigma ^3 \sqrt[]{n}} > 1$ qu'il n'y ait pas de solution. \\
Si $ \rho = 10, \sigma =1, n = 10 000, 1-\alpha = 0.95 $ : 
\[
    \phi (a) = 1 - 0.025 + 0.05 = 1.025 > 1 \text{impossible }
.\]

\underline{Nouveau cours du 18/11} \\

\begin{cor}[Inégalité de Berry-Esseen]
    Sous les hypothèses de l'inégalité de Berry-Esseen, on a : 
    \[
        \forall a \in \mathbb{R}^+, P(-a \leq Z_n \leq a) \geq P(-a \leq Z \leq a) - \frac{\rho }{\sigma ^3 \sqrt[]{n}}
    .\]
    Avec $ Z \sim \mathcal{N}(0,1) $ 

    \begin{proof}[Preuve: ]
        La démonstration a été faites le cours précèdent.
    \end{proof}
    
    
\end{cor}
Interprétation : 

Si $ a $ est tel que $ P(-a \leq Z \leq a) = 1 - \alpha $  alors $ P(-a \leq Z_n \leq a) \geq 1 - \alpha - \frac{\rho }{\sigma ^3 \sqrt[]{n}} $. 

On en déduit alors le niveau \textbf{réel} de l'intervalle de confiance asymptotique. 

Lien entre $ P(-a \leq Z_n \leq A) $ et $ P(m \in ) $ 
\[
    P(-a \leq Z_n \leq a) = P(m \in [\bar{X_n} - \frac{a \sigma }{\sqrt[]{n}} ; \bar{X_n} + \frac{a \sigma }{\sqrt[]{n}}])
.\]



\section{Théorème central limite avec autonormalisation}
Le point faible de ce qu'on a vu précédemment est qu'on requiert la connaissance de la variance des variables aléatoires considérées.

Dans de nombreux exemples, la connaissance de l'espérance et de la variance sont équivalentes. 

Exemple : \begin{itemize}
    \item si $ X \sim \mathcal{B}(p), E(X) = p, Var(X) = p(1-p) $. Si je connais $ p(1-p) $, je connais $ p $. 
    \item Si $ X \sim \mathcal{E}(\lambda ), \lambda > 0, E(X) = \frac{1}{\lambda }, Var(X)= \frac{1}{\lambda ^2} $ Si $ Var(X) = \frac{1}{100} $ alors $ \lambda = 10 $ et on sait tout ...
    \item On vas pas dire qu'on connaît la variance si on connaît pas l'espérance :sob:
\end{itemize} 
Pour beaucoup d'applications, on ne peut pas connaître la variance des données et le TCL est inapplicable. Le théorème suivant est alors utilisé.

\begin{thm}[Théorème central limite avec autonormalisation]
    Soit $ (X_i)_{i \in \mathbb{N}} $ une suite de variable aléatoire \textbf{i.i.d.} admettant un \textbf{moment d'ordre 2} (inconnu). De moyenne $ m=E(X_1) $  \\
    Soit $ V_n $ un estimateur \textbf{faiblement consistant} de $ Var(X_1) $ Alors :
    \[
        Z_n = \frac{\sqrt[]{n}}{\sqrt[]{V_n}}(\frac{1}{n}\sum_{i=1}^{n}X_i - m) \to^\mathcal{L}_{n \to +\infty } Z \text{ de loi } \mathcal{N}(0,1)
    .\]
\end{thm}
\begin{note}[]
    3 methode pour estimateur faiblement confisquant
    \begin{itemize}
        \item CVG presque sur 
        \item CVG en moyenne quadratique 
        \item A la main Tchebychev
    \end{itemize} 
\end{note}


\begin{rem}[]
    Les avantages : \begin{itemize}
        \item On peut faire des intervalles de confiance asymptotique ! Du type 
        \[
            []\bar{X_n} - \frac{1.96 \sqrt[]{V_n}}{\sqrt[]{n}}, \bar{X_n} + \frac{1.96 \sqrt[]{V_n}}{\sqrt[]{n}} ]
        .\]
    \end{itemize}
    Les désavantages : \begin{itemize}
        \item Pas d'inégalité de Berry-Esseen, donc pas d'IC, ni de correction
    \end{itemize}
\end{rem}

\begin{exmp}[Application aux sondages]
    On effectue un sondage dans la population, deux choix sont possibles : A et B. Les votens sont modélisé par des variables aléatoires i.i.d. de loi de Bernouilli de paramètre $ p $ inconnu \begin{align*}
        X_i = 1 \Leftrightarrow \text{le ième votant vote A} \\
        X_i = 0 \Leftrightarrow \text{le ième votant vote B} \\
    \end{align*}
    But : Intervalle de confiance asymptotique pour $ p $ de niveau $ 0.96 $ 
    \begin{enumerate}
        \item Choix de l'estimateur : On relie $ p $ au propriétées de $ X_i $. Comme $ E(X_i) = p $. On propose comme estimateur de $ p $ : $ \bar{X_n} = \frac{1}{n}\sum_{i=1}^{n}X_i $. Exo: montrer que c'est un estimateur sans biais et fortement conscistant de $ p $.
        \item On cherche un intervalle de confiance asymptotique de niveau 0.96 : \begin{itemize}
            \item Les varables $ (X_i)_{i \in \mathbb{N}} $ sont i.i.d. et admettent un moment d'ordre 2 (leurs variances vaut $ p(1-p) $ )
            \item Estimateur fortement consistent ? $ V_n = \frac{1}{n-1}\sum_{i=1}^{n}(X_i - \bar{X_n})^2 $ OU plus simple $ V_n = \bar{X_n}(1 - \bar{X_n}) $ \\
            Comme : \begin{itemize}
                \item $ \bar{X_n} \to ^{p.s} p  $ 
                \item $ x \mapsto x(1-x) continue $ 
            \end{itemize}
            Alors 
            \[
                \bar{X_n}(1 - \bar{X_n})
            .\]
            est aussi un estimateur fortement consistant pour $ Var(X_i) $ il est donc faiblement consistant.
        \end{itemize}
        Ainsi d'après le TCL avec autonormalisation appliqué à la suite de variable i.i.d. $ (X_i)_{i \in \mathbb{N}} $ admettant un moment d'ordre 2 et à un estimateur faiblement consistant de $ V_n^\prime = Var(X_1) $ on a : 
        \[
            \forall a \in \mathbb{R}^+, P(-a \leq \frac{\sqrt[]{n}}{\sqrt[]{V_n}} (\frac{1}{n}\sum_{i=1}^{n}X_i - p) \leq a) \to_{n \to +\infty } P(-a \leq Z \leq a) (Z \sim \mathcal{N} (0,1))
        .\]
        On cherche $ a $ dans la table tel que 
        \[
            P(-a \leq Z \leq a) = 0.96
        .\]
        C'est à dire 
        \[
            P(Z \leq a) = 0.98 \Leftrightarrow a = 2.06
        .\]
        Ainsi 
        \[
            P(-2.06 \leq frac{\sqrt[]{n}}{\sqrt[]{V_n}} (\frac{1}{n}\sum_{i=1}^{n}X_i - p) \leq 2.06)) \to 0.96
        .\]
        Or 
        \begin{align*}
                &-2.06 \leq frac{\sqrt[]{n}}{\sqrt[]{V_n}} (\frac{1}{n}\sum_{i=1}^{n}X_i - p) \leq 2.06 \\
            \Leftrightarrow & \frac{-2.06 \sqrt[]{V_n}}{\sqrt[]{n}} \leq \bar{X_n} - p \leq \frac{2.06 \sqrt[]{V_n}}{\sqrt[]{n}} \\
            \Leftrightarrow & \bar{X_n} - \frac{2.06 \sqrt[]{V_n}}{\sqrt[]{n}} \leq p \leq \bar{X_n} + \frac{2.06 \sqrt[]{V_n}}{\sqrt[]{n}}
        \end{align*}
        Ainsi 
        \[
            P(p \in [\bar{X_n} - \frac{2.06 \sqrt[]{V_n}}{\sqrt[]{n}}, \bar{X_n} + \frac{2.06 \sqrt[]{V_n}}{\sqrt[]{n}}]) \to 0.96
        .\]
        $ [\bar{X_n} - \frac{2.06 \sqrt[]{V_n}}{\sqrt[]{n}}, \bar{X_n} + \frac{2.06 \sqrt[]{V_n}}{\sqrt[]{n}}] $ est un intervalle de confiance asymptotique de niveau 0.96 pour $ p $ 
    \end{enumerate}
\end{exmp}


\underline{Nouveau cours du 07/12} \\
On a vu le théorème central limite avec autonormalisation. Malheureusement, ce résultat n'est qu'asymptotique. \\
Que faire si on veut quelque chose de précis ?

\section{Le cas des variables gaussienne}

\subsection{Rappel : variance connue}

$ X_1, \dots, X_n $ va. iid. de loi $ \mathcal{N}(m,\sigma ^2) $ avec $ m $ inconnu et $ \sigma ^2 $ connue. On a vu précédemment que 
\[
    \forall n \in \mathbb{N}^*, Z_n = \frac{\sqrt[]{n}}{\sigma } (\frac{1}{n}\sum_{i=1}^{n}X_i - m) \sim \mathcal{N}(0,1)
.\]
On obtient alors facilement des intervalles de confiance exacts et optimaux. \\

Que faire si la variance est inconnu ? 

\subsection{Cas de la variance inconnue}

$ X_1, \dots, X_n $ va. iid. de la $ \mathcal{N}(m, \sigma ^2) $ avec $ m $ et $ \sigma ^2 $ inconnues. Dans toute cette section : 
\[
    V_n = \frac{1}{n-1}\sum_{i=1}^{n}(X_i - \bar{X_n})^2
.\]
On ne va considérer que cet estimateur. Tout ce qu'on va raconter dans la suite est faux pour un autre choix d'estimateur de la variance. Pourquoi ce modèle est-il important ? \begin{itemize}
    \item Pragmatiquement, parce que c'est le seule modèle sur lequel je sais faire tant de chose 
    \item Les gaussiennes sont partout. On modélise souvent de l'incertitude (erreur de mesure) par des gaussiennes. Les mesures expérimentales montrent souvent des fluctuation gaussiennes car on mesure des moyennes
\end{itemize}

\begin{thm}[de Cochran]
    Si $ X_1, \dots, X_n $ sont des va. iid. de loi $ \mathcal{N}(m, \sigma ^2) $ alors 
    \begin{enumerate}
        \item $ \bar{X_n} $ et $ V_n $ sont \textbf{indépendante} (contre instinctif car il y a $ \bar{X_n} $ dans $ V_n $, miracle gaussien)
        \item $ \frac{(n-1)V_n}{\sigma ^2} $ suit la loi $ \mathcal{X}^2 (n-1) $ 
    \end{enumerate}
    Cette loi a pour densité 
    \[
        f_{n-1} (x) = \frac{1 }{2^{n-1/2} \Gamma (\frac{n-1}{2}) } x^{\frac{n-1}{2} - 1} e^{-x} \mathbbm{1}_{x>0}
    .\]
    où 
    \[
        \Gamma (\frac{n-1}{2}) = \int_{0}^{t} x^{\frac{n-1}{2} - 1} e^{-x} dx
    .\]
    est la fonction gamma d'Euler
\end{thm}


\subsection{Loi du khi deux}

Quelle est cette loi $ \mathcal{X}^2(n-1) $ ? \\
On dit que $ \mathcal{X}^2(n-1) $ est la loi du chi-deux à (n-1) degrés de liberté. \\
En général, la loi $ \mathcal{X}^2(h), h \in \mathbb{N} $ (h degrès de liberté) est la loi de $ \sum_{i=1}^{h}Y_i^2 $ où $ Y_i \sim \mathcal{N}(0,1) $ iid.
\[
    f_h(x) = \frac{1}{2^{h/2} \Gamma (\frac{h}{2})} x^{h/2 -1 } e^{-x} \mathbbm{1}_{x>0}
.\]
Si $ Z \sim \mathcal{X}^2(h) $ alors
\begin{align*}
    &E(Z) = E(\sum_{i=1}^{h}Y_i^2) = hE(Y_n^2) = h \\
    &Var(Z) = h Var(Y_i)
\end{align*}
\begin{figure}[!htbp]
    \centering
    \includegraphics*[width=.5\textwidth]{./figures/Chi-square_pdf.png}
    \caption{Loi du khi deux}
\end{figure}

\subsubsection{Intervalle de confiance à 0.95}

Comme c'est pas symétrique, il faut trouver $ a $ et $ b $ tel que $ P(Z \leq a) = 0.025 $ et $ P(Z \leq B) = 0.975 $ 

Utilisons le théorème de Cochran pour trouver un intervalle de confiance de niveau 0.95 pour $ \sigma ^2 $ 
\[
    \frac{(n-1) V_n}{\sigma ^2} \sim \mathcal{X}^2(n-1)
.\]
Je trouve $ a $ et $ b $ tq si $ Z \sim \mathcal{X}^2 (n-1) $ 
\[
    P(a \leq Z \leq b) = 0.95
.\]
On les trouve dans la table en lisant $ P(Z \leq a) = 0.025 $ et $ P(Z \leq b) = 0.975 $. Ainsi 
\begin{align*}
    & P(a \leq \frac{(n-1) V_n}{\sigma ^2} \leq b) = 0.95
    \Leftrightarrow & P(\frac{(n-1) V_n}{b} \leq \sigma ^2 \leq \frac{(n-1) V_n}{a}) = 0.95
\end{align*}
$ [\frac{(n-1) V_n}{b}, \frac{(n-1) V_n}{a}] $ est un intervalle de confiance de niveau 0.95 pour $ \sigma ^2 $.

\subsection{Théorème de Student}

\begin{thm}[de Student]
    $ X_1, \dots, X_n  $ va. iid. de loi $ \mathcal{N}(m, \sigma ^2) $ 
    \[
        \frac{\sqrt[]{n}}{\sqrt[]{V_n}} (\bar{X_n} - m) \sim t(n-1)
    .\]
    Loi de student à $ (n-1) $ degrés de liberté.
    \[
        t_{n-1}(x) = \frac{1}{\sqrt[]{(n-1) \pi }} \frac{\Gamma (\frac{n}{2})}{\Gamma (\frac{n-1}{2})} \frac{1}{(1 + \frac{t^2}{n-1}) ^{n/2}}
    .\]
\end{thm}
\begin{proof}[Preuve: ]
    $ T = \frac{Z}{V/h} $ student si $ Z \sim \mathcal{N}(0,1), V \sim \mathcal{X}^2(h) $. 

    On veut regarder 
    \begin{align*}
        &\frac{\sqrt[]{n}}{\sqrt[]{V_n}} (\bar{X_n} - m) = \frac{\sqrt[]{n}}{\sigma }(\bar{X_n} - m) * \frac{\sigma }{\sqrt[]{V_n}} 
    \end{align*}
    Trouvons la loi de $ \frac{\sqrt[]{V_n}}{\sigma } $
    \[
        \frac{\sqrt[]{V_n}}{\sigma } = \sqrt[]{\frac{V_n}{\sigma ^2}} = \sqrt[]{\frac{V_n (n-1)}{\sigma ^2}}_{(\sim \mathcal{X}^2(n-1))} * \frac{1}{\sqrt[]{n-1}}
    .\]
    $ \frac{\sigma }{\sqrt[]{V_n}} $ s'écrit comme $ \frac{1}{\sqrt[]{\mathcal{X}^2 (n-1) / (n-1)}} $ 

    Ainsi $ \frac{\sqrt[]{n}}{\sqrt[]{V_n} (\bar{X_n} - m )} $ est de la forme $ \frac{\mathcal{N}(0,1)}{\sqrt[]{\mathcal{X}^2 (n-1) / (n-1)}} $ et $ \bot $ par Cochran !
\end{proof}



\subsubsection{Loi de student}
    La loi de Student à $ k $ degrés de liberté, noté $ t(h) $ est la loi de 
    \[
        T = \frac{Z}{\sqrt[]{V/k}} \text{ où } Z \sim \mathcal{N}(0,1), V \sim \mathcal{X}^2 (h), Z \bot V
    .\]
    Elle a pour densité 
    \[
        t_h(x) = \frac{1}{\sqrt[]{h \pi }} \frac{\Gamma (\frac{h+1}{2})}{\Gamma (\frac{h}{2})} \frac{1}{(1 + \frac{t^2}{h})^{\frac{h+1}{2}}}
    .\]

    \begin{figure}[!htbp]
        \centering
        \includegraphics*[width=.5\textwidth]{./figures/Student_density.png}
        \caption{Loi de student}
    \end{figure}

\subsubsection{Intervalle de confiance}
IC pour m exact de niveau 0.95 à n fixé. $ \frac{\sqrt[]{n}}{\sqrt[]{V_n}} (\bar{X_n} - m ) \sim t(n-1) $ (thm de student). 

On cherche $ a_t \in \mathbb{R}$ tq si $ Z \sim t(n-1) $ on a 
\[
    P(-a_t \leq Z \leq a_t) = 0.95
.\]
Par symétrie de la loi de Student. Cela revient à trouver $ a_t $ tq 
\[
    P(Z \leq a_t) = 0.975
.\]
On trouve $ a_t $ dans la table de la loi $ t(n-1) $ et on a 
\[
    P(-a_t \leq \frac{\sqrt[]{n}}{\sqrt[]{V_n}} (\bar{X_n} -m ) \leq a_t) 0.95
.\]
On réécrit et on a 
\[
    P(m \in [\bar{X_n} \pm  \frac{\sqrt[]{V_n}}{\sqrt[]{n} a_t}])
.\]
\begin{rem}[]
    \textbf{Attention:} $ a_t $ est un quantile de la loi de Student à $ n-1 $ degrés de liberté. Ce n'est pas le "a" de la loi $ \mathcal{N}(0,1) $.

    Plus $ n $ est grand, plus $ a_t $ se rapproche du quantile de la gaussienne $ \mathcal{N}(0,1) $ 
\end{rem}

\section{Complément sur le cours}

\begin{thm}[de Slutsky]
    Si $ Z_n \to ^{\mathcal{L}} Z$  et $ Y_n \to ^{P} y \in \mathbb{R} (constante)$ alors 
    \[
        Y_n Z_n \to ^{\mathcal{L}} y Z
    .\]
    \begin{rem}[]
        En général $ Y_n = V_n $ estimateur faiblement consistant de la variance
    \end{rem}
\end{thm}

\textbf{TCL + Slutsky $\Rightarrow$ TCL autonormalisé} : 

$ (X_i)_{i \in \mathbb{N}} $ une suite de va. iid. et admettant un moment d'ordre 2, $ V_n $ estimateur faiblement consistant de $ \sigma ^2 = Var(X_1) $ 
\begin{align*}
    \frac{\sqrt[]{n}}{\sqrt[]{V_n}} (\bar{X_n} - m ) = \frac{\sqrt[]{n}}{\sigma} (\bar{X_n} - m) * \frac{\sigma }{\sqrt[]{V_n}}
\end{align*}
Comme 
\begin{align*}
    & Z_n = \frac{\sqrt[]{n}}{\sigma }(\bar{X_n} - m) \to ^\mathcal{L} Z \sim \mathcal{N}(0,1) \\
    & Y_n = \frac{\sigma }{\sqrt[]{V_n}} \to ^{\mathbb{P}} 1
\end{align*}
Alors 
\[
    Z_n Y_n = \frac{\sqrt[]{n}}{\sqrt[]{V_n}} (\bar{X_n} - m) \to ^\mathcal{L} 1 - Z = Z \text{ (d'après Slutsky) }
.\]

\begin{thm}[Méthode Delta]
    Soit $ (T_n)_{n \in \mathbb{N}} $ une suite de variables aléatoires et $ \theta \in \mathbb{R} $ tels que 
    \[
        \sqrt[]{n}(T_n - \theta ) \to ^{\mathcal{L}}_{n \to \infty } \mathcal{N}(0, \sigma ^2)
    .\]
    alors si $ g $ est une fonction dérivable telle que $ g'(\theta ) \neq 0 $, on a 
    \[
        \sqrt[]{n}(g(T_n) - g(\theta )) \to ^{\mathcal{L}}_{n \to \infty}  \mathcal{N}(0, \sigma ^2 g'(\theta)^2)
    .\]
    Ce théorème est très utile pour obtenir des intervalles de confiance asymptotique !
\end{thm}


\section{Récap du cours}
\begin{figure}[htbp]
    \centering
    \includegraphics*[width=0.75\textwidth]{./figures/Fiche recap BUTEZ P1.png}
\end{figure}

\begin{figure}[htbp]
    \centering
    \includegraphics*[width=0.75\textwidth]{./figures/Fiche recap BUTEZ P2.png}
\end{figure}

\end{document}