\documentclass{article}
\usepackage[utf8]{inputenc}
\usepackage[a4paper, margin=2.5cm]{geometry}
\usepackage{graphicx}
\usepackage[french]{babel}

\usepackage[default,scale=0.95]{opensans}
\usepackage[T1]{fontenc}
\usepackage{amssymb} %math
\usepackage{amsmath}
\usepackage{amsthm}
\usepackage{systeme} %use with \system*{eq1, eq2}
\usepackage{bbm}
\usepackage{tikz-cd}

\usepackage{import}
\usepackage{xifthen}
\usepackage{pdfpages}
\usepackage{transparent}



\newcommand{\incfig}[2][1]{%
	\def\svgwidth{#1\columnwidth}
	\import{./figures_chap1/}{#2.pdf_tex}
}

\usepackage{hyperref}
\hypersetup{
	colorlinks=true,
	linkcolor=blue,
	filecolor=magenta,      
	urlcolor=cyan,
}
\urlstyle{same} %\href{url}{Text}

\theoremstyle{plain}% default
\newtheorem{thm}{Théorème}[section]
\newtheorem{lem}[thm]{Lemme}
\newtheorem{prop}[thm]{Proposition}
\newtheorem*{cor}{Corollaire}
%\newtheorem*{KL}{Klein’s Lemma}

\theoremstyle{definition}
\newtheorem{defn}{Définition}[section]
\newtheorem{exmp}{Exemple}[section]
%\newtheorem{xca}[exmp]{Exercise}

\theoremstyle{remark}
\newtheorem*{rem}{Remarque}
\newtheorem*{note}{Note}
%\newtheorem{case}{Case}



\title{Topologie}
\author{Charles Vin}
\date{2021}

\begin{document}
\maketitle

\section{Espaces métriques}
\begin{defn}[Distance]
    Une distance sur l'ensemble $ E $ est une application $ d: E \times E \to [0;+\infty ], (x,y) \mapsto d(x,y) $ telle que pour tout $ (x,y) \in E^2, x,y,z \in E $ 
    \begin{itemize}
        \item $ d(x,y) = d(y,x)$ (symétrie)
        \item $ d(x,y) = 0 \Leftrightarrow x=y $ (séparation)
        \item $ d(x,z) \leq d(x,y) + d(y,z) $ (inégalité triangulaire)
    \end{itemize}
    On dit que $ (E, d) $ est un \textbf{espace métrique}.
\end{defn}
\begin{rem}[]
    L'inégalité triangulaire implique : 
    \[
        \left| d(x,y) - d(x,z) \right| \leq d(y,z)
    .\]
\end{rem}


\begin{defn}[Boules]
    La boule ouverte de rayon $ r>0 $ et de centre $ x \in E$ est $ B(x,r) = \{y \in E, d(x,y) < r\} $. \\
    La boule fermé de rayon $ r>0 $ et de centre $ x \in E$ est $ \bar{B}(x,r) = \{y \in E, d(x,y) \leq  r\} $. \\
\end{defn}

\begin{exmp}[Exemple de distance]
    \begin{enumerate}
        \item Distance grossière (existe sur tout ensemble E) : $ d(x,y) = \mathbbm{1}_{x \neq y} $ la boule ouverte : $ B(x,r) = \systeme*{
            \{x\} \text{ si } r < 1 ,
            E \text{ si } r \leq 1
        }$
        \item Distance euclidienne (distance usuelle) sur $ \mathbb{R} $ : $ d(x,y) = \left| x-y \right|  $ la boule ouverte $ B(x,r) = ]x-r;x+r[ $ 
        \item Distance euclidienne sur $ R^d $ : $ d(x,y) = \sqrt[]{\sum_{i=1}^{d}(x_i - y_i)^2} $ où $ x=(x_1, \dots, x_d), y=(y_1, \dots, y_d) $ boule ouverte = cercle
        \item Distance $ l^1 $ dans $ \mathbb{R}^d $ (distance de Manhattan (car toute les rues se coupes en angle droit)) : $ d_1 (x,y) = \sum_{i=1}^{d} \left| x_i - y_i \right| $ la boucle ouverte donne un losange.
        \item Distance $ l^p $ sur $ \mathbb{R}^d, (p \geq 1)$ : $ d_p (x,y) = (\sum_{i=1}^{d} \left| x_i - y_i \right| )^{1/p}, (0 < p < 1) : d_p (x,y) = (\sum_{i=1}^{d} \left| x_i - y_i \right| )^p$
        \item Distance $ l^\infty  $ sur $ \mathbb{R}^d $ : $ d(x,y) = \max (\left| x_1-y_1 \right| , \dots, \left| x_d -y_d \right| ) $ Les boules ouvertes sont des carrées.
    \end{enumerate}
\end{exmp}

\begin{defn}[]
    Soit $ A $ est une partie de $ E $ et d'une distance sur $ E $ \begin{itemize}
        \item $ A $ est un \textbf{voisinage} de $ x \in E $ s'il existe $ r>0 $ tel que $ B(x,r) \subset A $ 
        \item $ A $ est dit \textbf{ouvert} s'il est voisinage de chacun de ses points : $ \forall x \in A, \exists r>0, B(x,r) \subset A $ 
        \item $ A $ est fermé si son complémentaire est ouvert : $ A^C = E \setminus A $ ouvert 
    \end{itemize}
    \begin{rem}[]
        \begin{itemize}
            \item On peut être à la fois ouvert et fermé : ex: $ \mathbb{R} $ pour la distance usuelle
            \item On peut n'être ni ouvert ni fermé ex: $ ]0;1] $ pas voisinage, $ ]0;1]^C = ]-\infty ;0] \cup ]1;+\infty ] $ pas voisinage de 0
        \end{itemize}
    \end{rem}
\end{defn}
\begin{defn}[]
    La \textbf{topologie} d'un espace métrique est l'ensemble de ses ouverts. 
\end{defn}
\begin{prop}[]
    .
    \begin{enumerate}
        \item Les boules ouvertes sont des ouverts
        \item Les boules fermées sont des fermés
        \item Un voisinage peut être ni ouvert ni fermé
    \end{enumerate}
    \begin{proof}[Preuve: ]
        \begin{enumerate}
            \item Prenons $ B(x,r) $. Pour tout $ y \in B(x,r) $, $ B(x,r) $ est elle voisinage de $ y $. Faire le dessin. \\
            Il faut montrer que $ B(y, r-d(x,y)) \subset B(x,r) $, ça veut dire que tous les points de l'un sont dans l'autre $ \forall z \in B(y, r-d(x,y)) \Rightarrow z \in B(x,r) $. \begin{align*}
                & \forall z \in B(y, r_d(x,y)) \\
                & d(x,z) \leq d(x,y) + d(y,z) \\
                & \text{Or } d(y,z) < r-d(x,y) \\
                & d(x,z) < r \text{ donc } z \in B(x,r)
            \end{align*}
            \item Prenons $ \bar{B}(x,r) = \{y \in E, d(x,y) \leq r\} $
            $(\bar{B}(x,r))^C = \{y \in E, d(x,y) > r\}$ est-il ouvert ? \\
            Pour tout $ y \in (\bar{B}(x,r))^C : d(x,y) > r $ 
            \begin{align*}
                B(y, d(x,y)-r) &\subset (\bar{B}(x,r))^C \text{ car } \\
                \forall z \in B(y, d(x,y) -r ) &, d(y,z) < d(x,y) - r \\
                &\Leftrightarrow r < d(x,y) - d(y,z) \\
                \text{Or } d(z,x) \geq & \left| d(x,y) - d(y,z) \right| > r \\
                \text{On a bien } & d(z,x) > r, z \in (\bar{B}(x,r))^C
            \end{align*}
            \item $ ]0,1] $ n'est ni ouvert ni fermé dans $ \mathbb{R} $ (pour la distance usuelle) \\
            $ ]0;1] $ est voisinage de $ \frac{1}{2} $ car $ B(\frac{1}{2}, \frac{1}{10}) = ]\frac{4}{10}, \frac{6}{10} \subset ]0;1[ ?=? \{x \in \mathbb{R}, \left| x - \frac{1}{2} \right| < \frac{1}{10}\}$ 
        \end{enumerate}
    \end{proof}
    \begin{rem}[]
        Dans $ \mathbb{R} $ muni de la distance $ (x,y) \mapsto \left| x-y \right|  $ les boules ouvertes sont les intervalles ouverts bornés : 
        \begin{align*}
            & B(a,r) = ]a-r;a+r[ \\
            \forall \alpha < \beta, & ]\alpha , \beta [ = B(\frac{\alpha + \beta }{2}, \frac{\beta - \alpha }{2})
        \end{align*}
    \end{rem}
\end{prop}

\begin{prop}[]
    \begin{enumerate}
        \item $ E $ et $ \varnothing  $ sont ouverts et fermés
        \item Une union quelconque d'ouvert est un ouvert 
        \item Une intersection quelconque de fermés est un fermé
        \item Une intersection finie d'ouvert est un ouvert 
        \item Une union finie de fermés est un fermé
    \end{enumerate}
    \begin{proof}[Preuve:]
        \begin{enumerate}
            \item $ \forall x \in E, B(x,r) \subset E $  donc $ E $ est ouvert et $ \varnothing  $ est fermé. \\
            $ \forall x \in \varnothing, B(x,r) \subset \varnothing  $  donc $ E $ est ouvert et $ E $ est fermé. Car y'a rien dans $ \varnothing  $, car le vide n'a pas d'élément.
            \begin{note}[]
                $ E = ]0,1], d:(x,y) \mapsto \left| x-y \right|  $ \\
                $ B_{(]0;1], d)} (1, \epsilon ) = \{x \in ]0,1], \left| x-1 \right| < \epsilon \} = ]1-\epsilon ;1]$ 
                $ B_{(\mathbb{R}, d)} (1, \epsilon ) = \{x \in \mathbb{R}, \left| x-1 \right| < \epsilon \} = ]1-\epsilon ;1+\epsilon ]$ 
            \end{note}
    
            \item Prenons $ (A_i)_{i \in I} $ famille d'ouverts de $ (E,d) $\\
            $ \forall x \in \bigcup_{i \in I}^{}A_i, B(x,y) ?\subset?\bigcup_{i \in I}^{A_i}$ 
            \begin{align*}
                \forall x \in \bigcup_{i \in I}^{}A_i &\Rightarrow  \exists i \in I, x \in A_i \\
                & \Rightarrow \exists r > 0, B(x,r) \subset A_i \subset \bigcup_{j \in I}^{}A_j \text{ car } A_i \text{ ouvert}
            \end{align*}
    
            \item Prenons $ (F_i)_{i \in I} $  famille de fermés de $ (E,d) $ 
            \begin{align*}
                (\bigcap_{i \in I}^{}F_i)^C = \bigcup_{i \in I}^{}(F_i)^C \text{ Ouvert d'après 2}
            \end{align*}
            
            \begin{rem}[]
                2) et 3) marche pour $ I=\{1,2\}, I=\mathbb{N}, I=\mathbb{R} $ ou tout autre ensemble d'indice $ I $.
            \end{rem}
            
            \item Prenons $ (A_i)_{i \in [1,n]} $ une famille finie d'ouverts de $ (E,d) $. 
            \begin{align*}
                \forall x \in \bigcap_{i=1}^{n}A_i, & \forall i \in \{1,\dots, n\}, x \in A_i \text{ ouvert} \\
                & \exists r_i > 0, B(x,r_i) \subset A_i\\ 
                & B(x, \min (r_1, ..., r_n)) \subset \bigcap_{i=1}^{n}A_i \\
                \forall i \in \{1, ..., n\} & B(x, \min _{1 \leq i \leq n} r_i) \subset B(x,r_i) \subset A_i
            \end{align*}
    
            \item Prenon $ (F_i)_{1 \leq i \leq n} $ une famille finie de fermés de $ (E,d) $ 
            \begin{align*}
                (\bigcup_{i=1}^{n}F_i)^C = \bigcap_{i=1}^{n}(F_i)^C \text{ Ouvert d'après 4)}
            \end{align*}
        \end{enumerate}
    \end{proof}
\end{prop}
\begin{exmp}[Intersection infinie d'ouverts qui n'est pas un ouvert]
    $ \bigcap_{n=1}^{+\infty }]0 - \frac{1}{n}, 0 + \frac{1}{n}[ = \{0\} $ fermé, pas ouvert (il n'existe pas de $ \epsilon  $ tq. $ B(0, \epsilon ) = ]-\epsilon ; +\epsilon[ \subset \{0\} $  ) 
\end{exmp}

\begin{defn}[]
    Soit $ A $ une partie de $ (E,d) $ 
    \begin{enumerate}
        \item L'\textbf{intérieur} de $ A $ est l'ensemble des points de $ E $ dont $ A $ est voisinage 
        \[
            Int(A) = \dot{A} = \{x \in E, \exists r > 0 B(x,r) \subset A\} 
        .\]
        C'est le plus grand ouvert de $ E $ contenu dans $ A $.
        \item L'\textbf{adhérence} de $ A $ est l'ensemble des points de $ E $ dont tout voisinage intersecte $ A $ 
        \[
            Adh(A) = \bar{A} = \{x \in E, \forall r > 0 B(x,r) \cap A \neq \varnothing \}
        .\]
        C'est le plus petit fermé de $ E $ contenant $ A $ 
        \item on dit que $ A $ est \textbf{dense} dans $ E $ si $ \bar{A} = E $.
    \end{enumerate}
\end{defn}
\begin{proof}[Preuve que $ \dot{A} $ est ouvert: ]
    Soit $ x \in \dot{A} : \exists r^\prime > 0, B(x, r^\prime) \subset \dot{A} $ ?
    \begin{align*}
        \exists r > 0, & B(x,r) \subset A \\
                        & B(x, \frac{r}{2}) \subset \dot{A} \\
        \text{car }      & \forall y \in B(x, \frac{r}{2}); B(y, \frac{r}{2}) \subset B(x,r) \subset A
    \end{align*} 
    $ \dot{A} $ est voisinage de chacun de ses points 
\end{proof}
\begin{proof}[Preuve que $ \dot{A} $ plus grand ouvert inclus dans $ A $ : ]
    Si $ U $ ouvert avec $ U \subset A $
    \begin{align*}
        \forall x \in U, & \exists r>0 B(x,r) \subset u \subset A \\
                        & x \in \dot{A} \\
            U \subset \dot{A} &
    \end{align*}
\end{proof}


\section{Suites et limites}
\textbf{Rappel:} Une application définie sur $ \mathbb{N} $ s'appelle une suite. On l'appelle souvent :
\begin{align*}
    u: & \mathbb{N} \to E \\
        & n \mapsto u(n) = u_n
\end{align*}
On la note souvent $ (u_n)_{n \in \mathbb{N}} $. L'ensemble des suites à valeurs dans $ E $ est noté $ E^{\mathbb{N}} $.

\begin{defn}[]
    Dans l'espace métrique $ (E,d) $ la suite $ (u_n)_{n \in \mathbb{N}} \in E^{\mathbb{N}}$ \textbf{converge vers la limite $ l \in E$ } si : 
    \[
        \forall \epsilon > 0, \exists n_\epsilon \in \mathbb{N} \forall n \geq n_\epsilon, d(u_n, l) < \epsilon \Leftrightarrow u_n \in B(l,\epsilon )
    .\]
    On note $ \lim_{n \to \infty} u_n = l $ ou $ u_n \to _{n \to +\infty }^d l$ 
\end{defn}

\begin{rem}[]
    $ u_n \to _{n \to +\infty }^d \Leftrightarrow \lim_{n \to \infty} d(u_n, l) = 0  $ 
\end{rem}

\begin{prop}[]
    Dans un espace métrique, si la limite d'une suite existe, elle est unique

    \begin{proof}[Preuve: ]
        Si $ l $ et $ l' $ sont limites de $ (x_n)_{n \in \mathbb{N}} \in E^{\mathbb{N}^*}$ 
        \begin{align*}
            \forall \epsilon > 0, &\exists n_\epsilon, \forall n \geq n_\epsilon, d(u_n, l) < \epsilon \\
            \forall \epsilon > 0, &\exists n_\epsilon^\prime  \forall n \geq n_\epsilon^\prime , d(u_n, l^\prime ) < \epsilon \\
                    &\forall n \geq \max (n_\epsilon , n_\epsilon ^\prime ), 0 \leq d(l,l') \leq d(l,x_n)+d(x_n,l')< 2 \epsilon 
        \end{align*}
        donc $ d(l,l')=0 $ et $ l=l' $ 
    \end{proof}
\end{prop}

\begin{prop}[]
    Soit $ A $ une partie de $ (E,d) $ 
    \begin{enumerate}
        \item $ x \in \bar{A} \Leftrightarrow$ il existe une suite d'élément $ (x_n)_{n \in \mathbb{N}} $  de $ A $ qui converge vers $ x $.
        \item $ A $ est fermé $ \Leftrightarrow $ toute suite convergente d'éléments de $ A $ a sa limite dans $ A $.
    \end{enumerate}
    \begin{proof}[Preuve: ]
        \begin{enumerate}
            \item $ \Rightarrow $ | \begin{align*}
                x \in \bar{A} &\Leftrightarrow \forall r>0, &B(x,r) \cap A \neq \varnothing \\
                    &\Rightarrow \forall n \in \mathbb{N}, &B(x, \frac{1}{n}) \cap A \neq \varnothing \\
                    & &\exists x_n \in A, d(x,x_n)<\frac{1}{n} \\ 
                    \Leftrightarrow \lim_{n \to \infty} x_n = x
            \end{align*}
            $ \Leftarrow  $ | On suppose q'il existe $ (x_n)_{n \in \mathbb{N}} \in A^{\mathbb{N}}$ telle que $ x_n \to _{n \to +\infty }^d x$ 
            \begin{align*}
                \forall \epsilon > 0, \exists n_\epsilon \in \mathbb{N}, \forall n \geq n_\epsilon, &d(x_n,x) < \epsilon \\
                    & x_n \in B(x,\epsilon ) \\
                    & x_n \in B(x, \epsilon) \cap A \neq \varnothing \\
                    & \Leftrightarrow x \in \bar{A}
            \end{align*}
            \item $ A $ fermé $ \Leftrightarrow \bar{A} = A$ et d'après 1) $ \bar{A} = \{\text{Limites de suites d'éléments de } A\} $ 
        \end{enumerate}
    \end{proof}
\end{prop}

\begin{defn}[]
    .\\
    \begin{itemize}
        \item Le \textbf{diamètre} de la partie $ A $ de $ E $ est $ diam(A) = \sup \{d(x,y), x \in A, y \in A\} \in [0,+\infty]$ 
        \item $ A $ est dit \textbf{borné} si $ diam(A)<+\infty $ 
    \end{itemize}
\end{defn}

\begin{prop}[]
    $ A $ borné $ \Leftrightarrow \exists x \in E, \exists r>0, A \subset B(x,r) $ 
    \begin{proof}[Preuve: ]
        Si $ x \in A, r > diam(A) $ : 
        \[
            \forall y \in A, d(y,x) \leq diam(A) < r \Leftrightarrow y \in B(x,r)
        .\]
    \end{proof}
\end{prop}


\section{Suite de Cauchy, espace complet}


\begin{defn}[Suite de Cauchy]
    $ (x_n)_{n \in \mathbb{N}} $ est une \textbf{suite de Cauchy} dans l'espace métrique $ (E,d) $ si 
    \[
        \forall \epsilon > 0, \exists n_\epsilon \in \mathbb{N}, \forall k,l \geq n_\epsilon, d(x_k, x_l) < E
    .\]
    \begin{note}[]
        Les termes sont aussi proche qu'on veut les uns des autres au-delà d'un certain rang
    \end{note}
\end{defn}

\begin{prop}[]
    Toute suite convergente est de Cauchy
    \begin{proof}[Preuve: ]
        Si $ x_n \to _{n \to +\infty }^d x $ 
        \begin{align*}
            \forall \epsilon > 0, \exists n_{\epsilon /2} \in \mathbb{N}, \forall k \geq n_{\epsilon /2}, &d(x_k, x) < \frac{\epsilon }{2} \\
                        \forall l \geq n_{\epsilon /2}, &d(x_l, x) < \frac{\epsilon }{2} \\
                            & d(x_k, x_l) \leq d(x_k, x) + d(x,x_l) < \frac{\epsilon }{2} + \frac{\epsilon }{2} = \epsilon 
        \end{align*}
    \end{proof}
\end{prop}

\begin{rem}[]
    Il existe des suite de Cauchy non convergentes
    
    \begin{exmp}[]
        Sur $ \mathbb{Q} $ muni de la distance $ (x,y) \mapsto \left| x-y \right|  $. $ x_n = \sum_{j=0}^{n}\frac{1}{j!} $ on sait que $ \lim_{n \to \infty} \sum_{j=n}^{+\infty }\frac{1}{j!} = 0 $ c'est à dire : 
        \[
        \forall \epsilon > 0, \exists n_\epsilon \in \mathbb{N}, \forall n \geq n_\epsilon, \left| \sum_{j=0}^{+\infty }\frac{1}{j!} \right| < \epsilon 
        .\]
    Regardons pour $ x_n $ avec $ k \leq l $ 
    \[
        \forall k,l \geq n_\epsilon, \left| x_l - x_k \right| = \left| \sum_{j=k}^{l} \frac{1}{j!}\right| \leq \sum_{j=k}^{+\infty } \frac{1}{j!} \leq \epsilon 
        .\]
    \end{exmp}
    Mais $ (x_n)_n $ n'a pas de limite dans $ \mathbb{Q} $ car $ \sum_{j=0}^{+\infty }\frac{1}{j!} = e^1 \not\in  \mathbb{Q}$. \\
    \textbf{Intuitivement :} Une suite de Cauchy est une suite qui "devrait converger" si $ E $ avait "assez de points". 
\end{rem}

\begin{defn}[Espace complet]
    L'espace métrique $ (E,d) $ est \textbf{complet} si dans $ E $ toute suite de Cauchy converge.
    
    Pour la distance usuelle $ (x,y) \mapsto \left| x-y \right| $ \begin{itemize}
        \item $ \mathbb{R} $ est complet (admis).
        \item $ \mathbb{Q} $ n'est pas complet et $ \bar{\mathbb{Q}} = \mathbb{R} $ 
        \item $ ]0;1[ $ n'est pas complet. $ \{\frac{1}{n}, n \in \mathbb{N}^*\} $ n'est pas complet (car $ \lim_{n \to \infty} \frac{1}{n} = 0 $ ) 
    \end{itemize}
    Pour la distance euclidienne, $ \mathbb{R}^d $ est complet.
\end{defn}

\section{Compacité}

\begin{defn}[sous-suite]
    Une \textbf{sous-suite} ou \textbf{suite extraite} de la suite $ (x_n)_{n \in \mathbb{N}} $ est une suite obtenue ne gardant que certains des $ x_n $, pour une infinité d'indice $ n $, et en les renumérotant : 
    \[
        (x_{\varphi (n)})_{n \in \mathbb{N}} \text{ où } \phi : \mathbb{N} \to \mathbb{N} \text{ strictement croissante}
    .\]
    \begin{align*}
        x_0, \not x_1, x_2, \not x_3, \not x_4, x_5, \dots \\
        x_{\phi (0)} = x_0, x_{\phi (1)} = x_2, x_{\phi (3)} = x_5 \\
        \phi (0) = 0, \phi (1) = 2, \phi (3)=5, \dots
    \end{align*}
\end{defn}

\begin{prop}[]
    Dans l'espace métrique $ (E,d) $, $ (x_n)_n $ converge vers $ l $ si et seulement si toute sous-suite de $ (x_n)_n $ converge vers $ l $ 

    \begin{proof}[Preuve: ]
        Si $ x_n \to _{n \to +\infty }^d l : \forall \epsilon, \exists n_\epsilon \in \mathbb{N}, \forall n \geq n_\epsilon, d(x_n, l) < \epsilon  $ \\
        Si $ (x_{\phi (n)})_n$ extraite de $ (x_n)_n $ en prenant $ n_\epsilon ^\prime  $ tel que $ \phi (n_\epsilon ^\prime ) $ 
        \begin{align*}
            \forall \epsilon > 0, \exists n_\epsilon ^\prime \in \mathbb{N}, &\forall n \geq n_\epsilon ^\prime, d(x_{\phi (n)}, l ) < \epsilon \\
                & \phi (n) \geq \phi (n_\epsilon ^\prime ) = n_\epsilon 
        \end{align*}
        Réciproquement : Si $ (x_n)_{n \in \mathbb{N}} $ ne converge pas vers $ l $ : 
        \begin{align*}
            \exists \epsilon > 0, \forall n_\epsilon \in \mathbb{N}, \exists n \geq  n_\epsilon d(x_n, l) < \epsilon 
        \end{align*}
        \[
            I = \{j \in \mathbb{N}, d(x_j,l) \geq \epsilon \} \text{ est infini}
        .\]
        En renumérotant, $ (x_j)_{j \in I} $ on obtient une suite extraite de $ (x_n)_n $ qui ne converge pas vers $ l $ 
    \end{proof}
\end{prop}

\underline{Nouveau cours du 22/11} \\

\begin{defn}[Compacité]
    \begin{itemize}
        \item L'espace $ (E,d) $ est \textbf{compact} si toute suite de $ E $ admet (au moins) une sous-suite convergente.
        \item $ A $ est une partie compacte de $ (E,d) $ si toute suite à valeurs dans $ A $ admet une sous-suite qui converge dans $ A $ (la limite doit être dans $ A $ )
    \end{itemize}
\end{defn}

\begin{prop}[]
    \begin{enumerate}
        \item Toute partie compacte d'un espace métrique est fermée et bornée
        \item Dans $ \mathbb{R} $ ou $ \mathbb{R}^d $ muni de la distance euclidienne, la réciproque est vrais : les parties fermés et bornés sont compactes
    \end{enumerate}

    \begin{proof}[Preuve: ]
        \begin{enumerate}
            \item Soit $ A $ compact dans $ (E, d) $. \begin{itemize}
                \item Fermé ? $ A $ fermé $ \Leftrightarrow $ Toute suite convergence d'éléments de $ A $ a sa limite dans $ A \Leftrightarrow A = \bar{A}$ : $ \{\text{Limite de suites de } A^N\} \subset, \supset A $ \\
                Si $ (x_n)_n \in A^{\mathbb{N}^*} $ converge vers $ l \in E : (x_n)_n$ a une sous-suite qui converge vers un $ a \in A $ (par compacité) et $ (x_n)_n $ converge vers $ l : l = a \in A$ 
                
                \item Borné ? Par l'absurde, si $ A $ n'était pas borné : $ diam(A) = + \infty  $ 
                \[
                    \forall n \in \mathbb{N}, \exists x_n, y_n \in A, d(x_n,y_n) \geq n
                .\]
                $ (x_n)_n $ a une suite extraite $ (x_{\phi (n)})_n $ qui converge vers $ l \in A $ ($ x_0, x_1, x_2, \not x_3, \not x_4, x_5, \dots \\ $ ))\\
                $ (y_\phi (n))_n $ a une suite extraite $ (y_{\psi  (n)})_n $ qui converge vers $ l' \in A $ ( $ y_0, \not y_1, y_2, \not y_3, \not y_4, \not y_5, \dots \\ $ ) \\
                
                \begin{note}[]
                    On veut garder des mêmes termes 
                \end{note}
                \begin{align*}
                    \forall n \in \mathbb{N}, n \leq \psi (n) \leq d(x_{\psi (n)}, y_{\psi (n)}) & \leq d(x_{\psi (n)}, l) + d(l,l') + d(l', y_{\psi (n)}) \\ 
                    & \to_{n \to +\infty } 0 + (d(l,l') < +\infty) + 0
                \end{align*}
                
                Impossible 
                \begin{note}[]
                    Car on a à gauche tous les entiers et à droite un entier finis, on a trouver un majorant de tous les entiers, ça n'existe pas. 
                \end{note}
            \end{itemize}

            \item Soit $ F $ borné sur $ \mathbb{R} $ : $ \exists a<b $  réel $ F \subset [a,b] $ Soit $ (x_n)_n \in F^{\mathbb{N}} $ \\
            Un au moins des intervalles $ [a, \frac{a+b}{2}] $ et $ [\frac{a+b}{2}, b] $ contient les $ x_n $ pour une infinité d'indices $ n $, on le note $ J_1 $. \\
            Une au moins des moitiés de $ J_1 $  contient les $ x_n $ pour une infinité de $ n $, on la note $ J_2 $\\
            $ J_3 $ contient les $ x_n $ pour une infinité de $ n $ ect. \\
            Chaque $ J_k $ est de longueur $ \frac{b-a}{2^k} $

            On note \begin{align*}
                & \varphi (1) = \inf \{n \in \mathbb{N}, x_n \in J_1\}
                & \varphi (2) = \inf \{n \in \varphi (1), x_n \in J_2\}
                & \varphi (3) = \inf \{n \in \varphi (2), x_n \in J_3\}
            \end{align*}
            $ (x_{\varphi (n)})_n $ est une sous-suite de $ (x_n)_n $ telle que  $ \forall n, x_{\varphi (n)} \in J_n $ et même $ \forall k \geq n, x_{\varphi (k)} \in J_n $ 
            \[
                \forall k,l \geq n \left| x_{\phi (k)} (\in J_n) - x_{\phi (l)} (\in J_n) \right| \leq \frac{b-a}{2^n}
            .\]
            $ (x_{\phi (n)})_n $ est une suite de Cauchy dans $ \mathbb{R} $ qui est complet donc $ (x_{\phi (n)}) $ converge. De plus si $ F $ fermé alors $ \lim_{n \to \infty} x_{\phi (n)} \in F $. \\
            CCL ; Si $ F $ fermé borné de $ \mathbb{R} $, toute suite dans $ F $ a une sous-suite qui converge dans $ F $.            
        \end{enumerate}
    \end{proof}
\end{prop}

\begin{exmp}[de compacts de $\mathbb{R}$]
    \[
        [0,1], [a,b], \{0\}, [-3, -1] \cup [100,101], \{n \in \mathbb{Z}, \left| n \right| \leq 1000\}
    .\]
\end{exmp}
\begin{exmp}[de non compacts de $ \mathbb{R} $ ]
    \[
        ]0,1], ]a,b[, \{\frac{1}{n}, n \in \mathbb{N}^*\}, \mathbb{Q} \cap [3,10] \text{ non fermé}
    .\]
    \[
        \mathbb{N}, ]-\infty ; a], \mathbb{R} \text{ non bornés}
    .\]
\end{exmp}

\section{Continuité}

\begin{defn}[Continuité]
    Soient $ (E_1, d_1), (E_2, d_2) $ deux espaces métriques. L'application $ f: E_1 \to E_2 $ est \textbf{continue en $ x_0 \in E_1 $} si 
    \[
        \forall \epsilon > 0, \exists \delta > 0, \forall x \in E_1, d_1(x, x_0) < \delta \Rightarrow d_2(f(x), f(x_0)) < \epsilon 
    .\]
    Elle est \textbf{continue sur $ E_1 $} si elle est continue en $ x_0 $ pour tout $ x_1 $ de $ E_1 $. On note $ f \in C^0(E_1, E_2) $.
\end{defn}

\begin{exmp}[]
    \begin{itemize}
        \item Plein de fonctions continues sont déjà connues pour $ E_1 = E_2 = \mathbb{R} $ muni de la distance usuelle. 
        
        \item La fonction "distance au point $ x_0 $" est $ C^0 (E, \mathbb{R}) $  
        \begin{align*}
            f : & E \to \mathbb{R} \\
                & x \mapsto f(x) = d(x, x_0)
        \end{align*}
        car $ d_2(f(x), f(y)) = \left| f(x) - f(y) \right| = \left| d_1(x, x_0) - d_1(y,x_0) \right| \leq d(x,y) $ prendre $ \delta < E $ 

        \item Sur $ \mathbb{R}^n $ muni de la distance euclidienne, les fonctions coordonnées sont continues $ C^0(\mathbb{R}^n, \mathbb{R}) $ :
        \begin{align*}
            & \mathbb{R}^n \to \mathbb{R}
            & x \mapsto x_i
        \end{align*}

        \item L'addition et la multiplication sont continues de $ \mathbb{R}^2 $ vers $ \mathbb{R} $ pour la distance euclidienne. 
    \end{itemize}
\end{exmp}

\begin{prop}[]
    Les propriétés suivantes sont équivalentes : \begin{enumerate}
        \item $ f $ est continue de $ (E_1,d_1) $ vers $ (E_2, d_2) $ 
        \item Pour tout ouvert $ U_2 $ de $ E_2 $, $ f^{-1}(U_2) $ est un ouvert de $ E_1 $ 
        \item Pour tout fermé $ F_2 $ de $ E_2 $, $ f^{-1} $ est un fermé de $ E_1 $ 
        \item Pour toute suite $ (x_n)_n \in E_1^{\mathbb{N}^*} $ convergente : $ \lim_{n \to \infty} f(x_n) = f(\lim_{n \to \infty} x_n) $ 
    \end{enumerate}

    \begin{proof}[Preuve: ]
        On veut montrer l'équivalence de tout : $ 1 \Rightarrow 2 \Rightarrow 3 \Rightarrow 4 \Rightarrow 1  $ comme ça on a une boucle. 
        \begin{itemize}
            \item $ 1 \Rightarrow 2 $ : Pour $ U_2 $ ouvert de $ E_2 $, montrons que $ f^{-1} (U_2) $ ouvert. \\
            Soit $ x \in f^{-1}(U_2) $ existe-t-il $ r>0 $ tel que $ B(x,r) \subset f^{-1}(U_2) $ ?
            \begin{align*}
                x \in f^{-1}(U_2) \Leftrightarrow f(x) \in U_2 \Rightarrow\\
                U_2 \text{ ouvert } \Rightarrow \exists \epsilon > 0, B(f(x), \epsilon ) \subset U_2 \Rightarrow\\
                f \in C^0(E_1, E_2) \Rightarrow \exists \delta > 0, \forall y \in E_1, & d_1(x,y) < \delta \Rightarrow d_2(f(x), f(y)) < \epsilon \\
                        & y \in B(x, \delta ) \Rightarrow f(y) \in B(f(x), \epsilon ) \subset U_2 \\
                        & y \in B(x, \delta ) \Rightarrow y \in f^{-1}(U_2) \\
                        & B(x, \delta ) \subset f^{-1}(U_2)
            \end{align*}

            \item $ 2 \Rightarrow 3 $ : Pour $ F_2 $ fermé dans $ E_2 $
            \begin{align*}
                f^{-1}(F_2) = \{x \in E_1, f(x) \in F_2\} &= E_1 \setminus \{x \in E_1, f(x) \not\in F_2 \} \\
                &= E_1 \setminus \{x \in E_1, f(x) \in E_2 \setminus F_2 \} \\
                &= E_1 \setminus f^{-1}(E_2 \setminus F_2)
            \end{align*}
            $ E_2 \setminus F_2 $ ouvert, $ f^{-1}(E_2 \setminus F_2) $ ouvert d'après 2, $ E_1 \setminus f^{-1}(E_2 \setminus F_2) $ est fermé

            \item $ 3 \Rightarrow 4 $ : Si $ x_n \to ^{d_1} _{n \to +\infty } l \in E_1$. Si on n'avait pas $ f(x_n) \to ^{d_2} _{n \to +\infty } f(l) $ 
            \begin{align*}
                \exists \epsilon > 0, \forall n \in \mathbb{N}, \exists k \geq n, & d_2(f(x_k) , f(l)) \geq \epsilon \\
                    & f(x_k) \not \in B(f(l), \epsilon )\\
                    & x_k \in f^{-1}(E_2 \setminus B(f(l), \epsilon ))
            \end{align*}
            $ B(f(l), \epsilon ) $ ouvert, $ E_2 \setminus B(f(l), \epsilon ) $ fermée, $ f^{-1}(E_2 \setminus B(f(l), \epsilon )) $ fermée d'après 3, les $ x_k $ forment une suite de limite $ l $. \\
            Donc 
            \begin{align*}
                & l \in f^{-1}(E_2 \setminus B(f(l), \epsilon )) \\
                & f(l) \in E_2 \setminus B(f(l), \epsilon ) \\
                & f(l) \not\in B(f(l), \epsilon) \text{ contradiction !}
            \end{align*}

            \item S'il y avait un $ x_0 \in E_1 $ en lequel $ f $ n'est pas continue 
            \[
                \exists x_0 \in E_1, \exists \epsilon > 0, \forall \delta > 0, \exists y \in E_1, d_1(x_0, y) < \delta \text{ et } d_2(f(x_0), f(y_n)) \geq \epsilon 
            .\]
            En prenant $ \delta = \frac{1}{n} $ 
            \[
                \exists y_n \in E, d(x_0, y_n) < \frac{1}{n} \text{ et } d_2(f(x_0) , f(y_n)) \geq \epsilon 
            .\]
            Ainsi 
            \[
                y_n \to ^{d_1} x_0 \text{ et } f(y_n) \to ^{d_2} f(x_0) \text{ contredirait 4 }
            .\]           
        \end{itemize}
    \end{proof}
\end{prop}

\begin{thm}[L'image continue d'un compact est compacte]
    Si $ f : E_1 \to E_2 $ continue et si $ A $ est une partie compacte de $ E_1 $, alors $ f(A) $ est une partie compacte de $ E_2 $ 

    \begin{proof}[Preuve: ]
        Si $ (y_n)_n $ suite dans $ f(A) = \{y \in E_2, \exists x \in E_1, f(x) = y \} $, $ \forall n \in \mathbb{N}, \exists x_n \in E_1, f(x_n) = y_n $. \\
        $ A $ compact $ \Rightarrow (x_n)_n $ a une sous-suite $ (x_{\varphi (n)})_n $ qui converge \\
        $ f $ continue, $ y_{\varphi (n)} = f(x_{\varphi (n)}) \to ^{d_2}_{n \to +\infty } f(\lim_{n \to \infty} x_n) \in f(A) $ 
    \end{proof}
\end{thm}

\begin{cor}[]
    Si $ f: E \to \mathbb{R} $ est continue et si $ E $ est compact alors $ f $ atteint ses borne sur $ E $ :
    \begin{align*}
        \exists x_+ , x_- \in E, & f(x_+) = \sup \{f(x), x \in E\} = \sup _E f\\
            & f(x_-) = \inf \{f(x), x \in E\} = \inf _E f 
    \end{align*}

    \begin{proof}[Preuve: ]
        $ f(E) $ est compact donc borné $ -\infty < \inf _E f \leq \sup _E f < +\infty  $ et fermé $ \sup _E f(E) \in f(E) $ idem pour inf. \\
    \end{proof}
    
    \textbf{Conséquence} : Dans un compact, les problèmes d'optimisation pour les fonctions continues ont des solutions ! 
\end{cor}

\underline{Nouveau cours du 29/11} \\

\section{Le Théorème du point fixe}

\begin{defn}[Fonction Lipschiztienne]
    $ f: E_1 \to E_2 $ est \textbf{lipschiztienne} s'il existe une constante $ K > 0 $ telle que 
    \[
        \forall x,y \in E_1, d_2(f(x), f(y)) \leq K d_1(x,y)
    .\]
    $ f $ est \textbf{contractante} si $ K < 1 $ 
\end{defn}
\begin{rem}[]
    Lipschitzienne $ \Rightarrow  $ continue (avec $ \delta = \frac{\epsilon }{K} $ )
\end{rem}

\begin{thm}[Du point fixe]
    Si $ (E,d) $ est un espace complet, toute application $ f : E \to E $ contractante admet un unique point fixe 
    \[
        \exists ! x \in E, f(x) = x
    .\]
\end{thm}

\begin{proof}[Preuve: ]
    \textbf{Unicité:} si $ x $ et $ y $ deux points fixes : $ d(x,y) = d(f(x), f(y)) \leq K d(x,y) \Rightarrow d(x,y) = 0 \Rightarrow x=y$ \\
    \textbf{Existence:} On fixe $ x_0 \in E $ et on note $ x_1 = f(x_0), x_2 = f(x_1), x_3 = f(x_2), ... $ 
    \[
        d(x_n, x _{n+1} ) = d(f(u _{n-1}, f(x_n) )) \leq K d(x _{n-1} , x_n), \forall n
    .\]
    donc $ \forall n \in \mathbb{N} $ 
    \[
        d(x_n, x _{n+1} ) \leq K^n d(x_1,x_0)
    .\]
    \begin{align*}
        d(x_n, x_k) & \leq d(x_n, x_{n+1}) + d(x _{n+1}, x _{n+2} + \dots + d(x_{k-1}, x_k) ) \text{ Par IT} \\
            & \leq (K^n + k^{n+1} + \dots + K^{k-1}) d(x_1, x_0)\\
            & \leq \frac{K^n -K^k}{1-K} d(x_1, x_0)
    \end{align*}
    Or $ \frac{K^n -K^k}{1-K} < \epsilon $ pour $ n,k \geq n_\epsilon  $ assez grand. \\
    La distance entre deux points peut être majoré pour tout $ \epsilon $. Donc $ (x_n)_{n} $ est de Cauchy dans $ (E, d) $ qui est complet, donc elle a une limite $ l \in E $
    \[
        d(l, f(l)) \leq d(l, x_n)_{} + d(x_n, f(x_n)) + d(f(x_n), f(l)) \to_{n \to +\infty } 0 + 0 + 0 \Leftrightarrow d(l, f(l)) = 0 \Leftrightarrow l=f(l)
    .\]
    
\end{proof}









\end{document}