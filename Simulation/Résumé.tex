\documentclass{article}
\usepackage[utf8]{inputenc}
\usepackage[a4paper, margin=2.5cm]{geometry}
\usepackage{graphicx}
\usepackage[french]{babel}

\usepackage[default,scale=0.95]{opensans}
\usepackage[T1]{fontenc}
\usepackage{amssymb} %math
\usepackage{amsmath}
\usepackage{amsthm}
\usepackage{systeme}
\usepackage{bbm}

\usepackage{hyperref}
\hypersetup{
    colorlinks=true,
    linkcolor=blue,
    filecolor=magenta,      
    urlcolor=cyan,
    pdftitle={Overleaf Example},
    % pdfpagemode=FullScreen,
    }
\urlstyle{same} %\href{url}{Text}

\newtheorem{thm}{Théorème}[]


\title{Tentative de résumé du cours de Simulation}
\author{Charles Vin}
\date{}

\begin{document}
\maketitle

\section{La fonction quantile}
Fonction quantile = inversion de la fonction de répartition
\begin{figure}[!htbp]
    \centering
    \includegraphics*[width=\textwidth]{fct_quantile.png}
    \caption{Pour gerer les sauts des truc discret}
\end{figure}

Méthode :
Je pense que y'a qu'en le faisant que j'arriverai il faut bien comprendre la formule et l'appliquer 
\begin{align*}
    F_x^-1 : &]0,1[ \longrightarrow \mathbb{R} \\ 
            & \alpha \mapsto F_X^-1(\alpha ) = \inf \{t \in R, F_X(t) \geq \alpha \}
\end{align*} 

Graphiquement
\begin{itemize}
    \item Sur le graphique de la fonction de répartition, on place $ \alpha $ sur l'ordonnée.
    \item On vas faire varier $ \alpha $ 
    \item Avec $ \alpha \in [..., ...] $ sur quelle intervalle $ F_X \geq \alpha  $ 
    \item Puis prendre l'$ \inf $, la borne inférieur de cette intervalle
\end{itemize}

On peut le faire analytiquement aussi quand $ F_X $ est continue strictement croissante (=bijective). Typiquement (pour loi $ \mathcal{E}xp $ )
\begin{align*}
    \forall \alpha \in ]0;1[, F_X^-1 &= \inf \{t \in R, F_X(t) \geq \alpha \} \\
            & = \inf \{t \in R^+_*, F_X(t) \geq x\} \\
            & = \inf \{t \in R^+_*, 1-e^{-\lambda t} \geq x\} \\
            & = \inf \{t \in R^+_*, t \geq \frac{\ln (1-\alpha  )}{-\lambda } \}  = -\frac{\ln (1-\alpha )}{\lambda }
\end{align*}

\section{Méthode du rejet }
$ f_X $ densité à simuler, $ c $ constante, $ g_X $ densité d'une va qu'on sait simuler.

\begin{thm}[]
    Soient $ f $ et $ g $ deux densité de probabilité, $ X $ de densité $ g $ , telles qu'il existe une constante $ c $ vérifiant :
    \[
        \forall x \in \mathbb{R}, f(x) \leq cg(x)
    .\]
    Si $ c*U_i*g(X_i) < f(X_i) $, X a pour densité $ f $ ($ U \sim Unif(]0,1[) $ ).\\
    $ U_i, X_i $ tirage des lois
\end{thm}
Le nombre de tirage nécessaire pour atteindre la condition précédente suit une loi $ Géom(\frac{1}{c}) $ 

Méthode pour les exos : 
\begin{itemize}
    \item J'pense dessiner la courbe c'est pas mal 
    \item Voir qu'on peut majorer easy par une constante + une loi uniforme 
    \item \textit{Exemple pour une densité avec $ max=\frac{1}{2} \Rightarrow  c*g = 2 * \frac{1}{4} \mathbbm{1}_{[0,4]} $ }
    \item Pour simuler \begin{itemize}
        \item On tire successivement V1, V2, V3,. . .indépendants tous de même loi que V , ce qui est facile en prenant Vi = ... où les Wi sont i.i.d. de loi uniforme sur ]0; 1[.
        \item On tire aussi successivement U1, U2, U3,. . .de loi uniforme sur ]0; 1[, indépendants entre eux et indépendants des Vi.
        \item On garde le premier Vi tel que
            \[
                c*f_V(V_i)*U_i \leq  f(V_i)
            .\]
    \end{itemize}
    \item Puis simplifier la condition avec les fonctions trouvé (car la loi uniforme tombe toujours dans l'intervalle des indicatrice)
    \item \textit{text}
\end{itemize}

\section{Simulation particulière de loi spécifiques}
Anki

\end{document}